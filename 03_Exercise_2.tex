\chapter{Exercise 2}

\section{Question 1}
We compute the optimal portfolio for an investor using Markowitz' formula. We assume that the investor seeks to maximize the following expected utility function:
\begin{equation*}
\max_\omega E\left[-e^{-\lambda X}\right]
\end{equation*}
where $X$ represents the expected returns and it is a random variable following the normal distribution with $N(\mu, \Sigma)$, where $\Sigma$ is the covariance matrix. Using the Laplace transform, we can rewrite it as

\begin{equation*}
\max_\omega E\left[-e^{{-\lambda \omega^T+}\frac{1}{2}\lambda^2\omega^T\Sigma\omega}\right]
\end{equation*}
From the first order conditions we get that the optimal weight allocation on the different investment possibilities is given by the vector
\begin{equation*}
\omega = \frac{1}{\lambda}\Sigma^{-1}\mu
\end{equation*}

\section{Question 2}

We have calculated the performance of the portfolio as $\mu^T\omega$ with variance $\omega^T\Sigma\omega$, which for our data is :

Expected return on the portfolio: 0.0536

Variance of the portfolio: 0.01788
\\
Figure \ref{fig4} shows the distribution of the performance.

\begin{figure}[ht]
\centering
\includegraphics[width=7cm, angle=270]{Q2_2plot.eps}
\caption{Expected performance of the optimal portfolio }
\label{fig4}
\end{figure}
