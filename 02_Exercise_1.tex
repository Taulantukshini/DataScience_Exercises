
\chapter{Exercise 1}

\section{Question 1}

\subsection{Compute the expectation and the variance of this distribution}
\subsubsection{Expectation}

\begin{align}
    \begin{split}
        P(X=k) =& \; \frac{\lambda^k e^{-\lambda}}{k!} \;\; ; \; X \sim P(\lambda) \\
        E[X] =& \; \sum_{k=0}^{\infty} k \cdot P(X=k) \\
        \Rightarrow \; E[X] =& \; \sum_{k=0}^{\infty} k \cdot \frac{\lambda^k \cdot e^{-\lambda}}{k!} \\
        \Rightarrow \; E[X] =& \; \sum_{k=1}^{\infty} \frac{\lambda^{k} e^{-\lambda}}{(k-1)!} \\
        \Rightarrow \; E[X] =& \; \lambda e^{-\lambda} \cdot \sum_{k=1}^{\infty} \frac{\lambda^{k-1}}{(k-1)!} \;\; ; \;\; \lim_{n \to \infty} \sum_{n}^{\infty} \frac{\lambda^n}{n!} = e^{\lambda} \\
        \Rightarrow \; E[X] =& \; \lambda e^{-\lambda} e^{\lambda} \\
        \Rightarrow \; E[X] =& \; \lambda
    \end{split}
\end{align}

As we can see, the estimator is the mean of the sample. The expected value of a Poisson random variable is equal to its paramenter $\lambda$, therefore the sample mean is an unbiased estimator of the expected value.

\subsubsection{Variance}

\begin{align}
    \begin{split}
        P(X=k) =& \; \frac{\lambda^k e^{-\lambda}}{k!} \;\; ; \; X \sim P(\lambda) \\
        Var[X] =& \; E[X^2] - E^2[X] \\
        \Rightarrow \; Var[X] =& \; \sum_{k=0}^{\infty} k^2 \cdot P(X=k) \; - \lambda^2 \\ \\
        E[X^2] =& \sum_{k=0}^{\infty} k^2 \cdot \frac{\lambda^k \cdot e^{-\lambda}}{k!} \\
        \Rightarrow \; E[X^2] =& \; \lambda e^{-\lambda} \cdot \sum_{k=1}^{\infty} k \cdot \frac{\lambda^{k-1}}{(k-1)!} \\
        \Rightarrow \; E[X^2] =& \; \lambda e^{-\lambda} \cdot \left( \sum_{k=1}^{\infty} (k-1) \frac{\lambda^{k-1}}{(k-1)!} + \sum_{k=1}^{\infty} \frac{\lambda^{k-1}}{(k-1)!} \right) \\
        \Rightarrow \; E[X^2] =& \; \lambda e^{-\lambda} \cdot \left( \lambda \cdot \sum_{k=1}^{\infty} \frac{\lambda^{k-2}}{(k-2)!} + \sum_{k=1}^{\infty} \frac{\lambda^{k-1}}{(k-1)!} \right) \;\; ; \;\; \lim_{n \to \infty} \sum_{n}^{\infty} \frac{\lambda^n}{n!} = e^{\lambda} \\
        \Rightarrow \; E[X^2] =& \; \lambda e^{-\lambda} \cdot (\lambda e^{\lambda} + e^{\lambda}) \\
        \Rightarrow \; E[X^2] =& \; \lambda^2 + \lambda \\ \\
        \Rightarrow \; Var[X] =& \; \lambda^2 + \lambda - \lambda^2 \\
        \Rightarrow \; Var[X] =& \; \lambda
    \end{split}
\end{align}

Again, the variance of a Poisson random variable is equal to its parameter $\lambda$ [blablabla efficiency bla....]

\section{Question 2}
There are several reasons, first of all $\lambda$ is the only parameter we need to define the Poisson distribution, once $\lambda$ is known then mean and variance follow effortlessly as seen in the previous exercise.
The Poisson distribution also possess another interesting and useful property: the Markov property (memorylessness). In probabilistic forecasting is a desirable property that may enable a simplified way to the solution of complex problems.[...]
The estimator $\lambda$ distributes exactly like a Poisson distribution with parameter n$\lambda$. This distribution, for a sufficient big n can be approximated by the Normal distribution with same average and variance as the relative Poisson distribution.

\section{Question 3}
As shown in the plot, the distribution is centered on 0.5 and we clearly see the impact of the number of observations on the 4th momentum (Kurtosis). In fact, the more the observations, the more efficient our estimator becomes, resulting on a peak in the density plot.
\section{Question 4}
Graph 4 shows the the impact of the number of observations on the estimator's volatility. The marginal change in speed of convergence is descending and we can reach a reasonably good measure (i.e. converge to the real value) with relatively few observations.
\section{Question 5}
Plots 5.1-4 show the comparison in distribution between the estimator and its gaussian approximation. We can clearly state that the convergence in distribution towards a Normal distribution is very quick and remains accurate even with a small number of observations. Again, the estimator itself is a random variable!
\section{Question 6}
